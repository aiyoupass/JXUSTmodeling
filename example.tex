\documentclass{JXUSTmodeling}

\makeatother

\biaoti{全国大学生数学建模竞赛 \LaTeX{} 模板使用手册}
\keyword{手册;数模;全国大学生数学建模竞赛;\TeX{};\LaTeX{}}
\begin{document}

\begin{abstract}
  为了让参加数学建模比赛的人专注内容,而尽可能少去关注格式,特此开发出此模板.以上是套话。

  LaTeX的好处的是让掌握他的人能以最省力最舒服的方式得到最规范的文档。当然他的工作方式并没有Microsoft Word那么简单直接(Word的特点是大家都能迅速上手但一般人需要花时间来让文档变得规范和漂亮)。这也导致他的学习曲线略有陡峭.
  
  模板代码托管地址:\url{https://github.com/aiyoupass/JXUSTmodeling/}。
\end{abstract}
\newpage
\tableofcontents

\section{使用例子}
\subsection{标题和关键字}
生成标题页
\begin{hexample}
\biaoti{标题}
\keyword{关键词;关键词}
\begin{abstract}
  关键词在这段文字之后
\end{abstract}
\end{hexample}
\subsection{数学公式}
西文字体和数学公式用了大家喜闻乐见的 Times 风格的 \texttt{newtxtext} 和 \texttt{newtxmath} 宏包
\begin{hexample}
Times
\[ \widehat{f}(\xi) = \frac{1}{(2\pi)^{d/2}}+ \int_{\mathbb{R}^d} e^{-ix \cdot \xi} f(x) \dd{x} \]
\end{hexample}

方便输入 Physical quantities 的 \texttt{siunitx} 宏包和简便数学公式输入的 \texttt{physics} 宏包
\begin{hexample}
\SI{123}{cm} \quad \si{Hz} \quad \num{3.45d-4}\par
$\int f(x) \dd{x}$ \quad $\dv{x}$
\end{hexample}

小于等于号的样式
\begin{hexample}
$\le = \leq = \leqslant$ \quad $\ge = \geq = \geqslant$
\end{hexample}
公式输入
\begin{hexample}
有编号行间公式
\begin{equation}\label{eq:example}
  E = mc^2
\end{equation}
无编号行间公式
\[ E = mc^2 \]
行内公式
$E = mc^2$\par
多行公式
\begin{align}
  E &= mc^2\\
  E/c^2 &= m
\end{align}
\end{hexample}
\subsection{表格和线表}
定义了 \mintinline{latex}{tabularx} 环境中的 \mintinline{latex}{Y} 选项
\begin{hexample}
\begin{tabularx}{\textwidth}{XY}
  \hline
  左对齐(X) & 局中对齐(Y)\\
  \hline
\end{tabularx}
\end{hexample}

三线表的输入
\begin{hexample}
\begin{tabularx}{\textwidth}{YY}
  \toprule
  文字 & 文字\\
  \midrule
  文字 & 文字\\
  文字 & 文字\\
  \bottomrule
\end{tabularx}
\end{hexample}
如果要添加竖线,建议用 \mintinline{latex}{\Xhline{线宽}} 制作三线表
\begin{hexample}
\begin{tabularx}{\textwidth}{c@{\hspace{1pc}}|@{\hspace{1pc}}X}
  \Xhline{0.08em}
  符号 & \multicolumn{1}{c}{符号说明}\\
  \Xhline{0.05em}
  $\int$ & 积分符号\\
  \Xhline{0.08em}
\end{tabularx}
\end{hexample}
表格尽量放在浮动体环境中
\begin{floatexample}
\begin{table}[htbp]
  \centering
  \caption{表注}\label{tab:example}
  \begin{tabularx}{\textwidth}{YY}
    \toprule %这是表格顶端的横线
    文字 & 文字\\
    \midrule %这是表格中间的横线
    文字 & 文字\\
    文字 & 文字\\
    \bottomrule % 这是表格下端的横线
  \end{tabularx}
\end{table}
\end{floatexample}

\subsection{插图和图注}
插图:图片建议放在 \texttt{figures} 文件夹里。基础使用方法
\begin{floatexample}
\begin{figure}[htbp]
  \centering
  \includegraphics[width=0.4\textwidth]{example-image.pdf}
  \caption{图注}\label{fig:example}
\end{figure}
\end{floatexample}
使用子图
\begin{floatexample}
\begin{figure}[htbp]
  \centering
  \subfloat[子图注]{\label{sub-fig-1}% 为子图加交叉引用
  \begin{minipage}{0.4\textwidth}
    \centering
    \includegraphics[width=\textwidth]{example-image-a}
  \end{minipage}
  }
  \qquad
  \subfloat[子图注]{%
  \begin{minipage}{0.4\textwidth}
    \centering
    \includegraphics[width=\textwidth]{example-image-b}
  \end{minipage}
  }
  \caption{图注}\label{fig:subexample}
\end{figure}
\end{floatexample}

\subsection{引用和代码}
交叉引用使用方法
\begin{hexample}
数据拟合图像见\ref{fig:example}。
结果图像见\ref{fig:subexample}
的子图~\ref{sub-fig-1}。
结果数据见\ref{tab:example}。
推导公式见 \ref{eq:example}。
\end{hexample}
用 \mintinline{latex}{\labelformat} 定制了 \mintinline{latex}{\ref} 的输出格式,无需考虑加“图”、“表”、“式”等字


参考文献引用,用知网/谷歌学术导出(这里并没有使用BibTeX)\\
使用方式是上谷歌学术网站(scholar.google.com)找到所需论文的引用
\begin{hexample}
文字\cite{label1}
文字\cite{label2}
\bibitem{label3}Newton I. Mathematical principles of natural philosophy[M]. A. Strahan, 1802.
\end{hexample}

用 \mintinline{latex}{listings} 宏包排版代码环境,定制了 MATLAB 环境和 Python 环境
\begin{hexample}
\begin{matlab}
clc,clear;
\end{matlab}
\begin{python}
print("Hello Python world!")
\end{python}
\end{hexample}

\begin{thebibliography}{99}
  \bibitem{label1}Einstein A. On the electrodynamics of moving bodies[J]. Annalen der physik, 1905, 17(10): 891-921.
  \bibitem{label2}Maxwell J C. The Scientific Letters and Papers of James Clerk Maxwell: Volume 1, 1846-1862[M]. CUP Archive, 1990.
  \bibitem{label3}Newton I. Mathematical principles of natural philosophy[M]. A. Strahan, 1802.
\end{thebibliography}



\begin{appendixx}
  \begin{hexample}
    \section{问题一的 Mathematica 代码}
    \begin{matlab}
      Solve[{2 a x + b + x/Sqrt[x^2 + y^2 + z^2] == 0, 
      2 a y + b + y/Sqrt[x^2 + y^2 + z^2] == 0, 
      b + z/Sqrt[x^2 + y^2 + z^2] == 0, x^2 + y^2 == 1, 
      x + y + z == 0}, {x, y, z, a, b}]
    \end{matlab}
  
    \section{问题二的 C++ 代码}
    \begin{minted}{c++}
#include <stdio.h>
int main () { 
  cout << "jetbrains";
} // 我是一个小笨蛋
    \end{minted}  
  \end{hexample}
  \section{问题一的 MATLAB 代码}
  \begin{matlab}
clc,clear;
  \end{matlab}

  \section{问题二的 C 代码}
  \begin{minted}{c++}
#include <stdio.h>
int main () { 
  printf("hi! the cruel world"); 
} // 我是一个大聪明
  \end{minted}
\end{appendixx}

\end{document}